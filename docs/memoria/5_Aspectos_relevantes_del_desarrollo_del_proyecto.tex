\capitulo{Aspectos relevantes del desarrollo del proyecto}
\label{cha:Aspectos relevantes del desarrollo del proyecto}

Este apartado contiene algunos aspectos relevantes sobre el desarrollo del
proyecto, incluyendo las razones detrás de las decisiones tomadas durante este y
el impacto que dichas decisiones han tenido sobre los resultados obtenidos.

\section{Fase de experimentación}

El conjunto de datos disponible durante la fase de experimentación consisten en
158 muestras con los siguientes ``atributos'':

\begin{itemize}
    \item Si la persona correspondiente a la instancia padece la enfermedad de
    Parkinson. Es lo que se busca predecir.
    \item Archivo de vídeo de una mano de la persona realizando la prueba de
    <<finger-tapping>>.
    \item Mano que aparece en el vídeo (izquierda o derecha).
    \item Fecha en la que se ha tomado el vídeo (irrelevante).
    \item Edad de la persona.
    \item Sexo de la persona.
    \item Mano dominante de la persona.
\end{itemize}

Cabe destacar que se ha tomado una muestra de cada mano de cada persona. Por lo
que este conjunto de datos ha sido obtenido de 79 personas. Aún así, se ha
tomado cada mano como una instancia distinta con el objetivo de tener más
instancias con las que trabajar.

Además, se dispone de 69 instancias de personas que padecen la enfermedad de
Parkinson y 89 instancias de personas sanas, por lo que se está trabajando con
un conjunto de datos ligeramente desbalanceado. Esto deberá tenerse en cuenta
para que no influya en los resultados obtenidos, por ejemplo, seleccionando
métricas adecuadas.

\subsection{Procesado de vídeo}

El objetivo de la fase de experimentación es crear un sistema que tenga por
entrada los atributos antes mencionados (salvo el objetivo) y por salida si esos
atributos corresponden con una persona que padece la enfermedad de Parkinson
(posiblemente junto con el grado de confianza de la predicción).

Los algoritmos de aprendizaje automático típicamente están diseñados para
trabajar con atributos numéricos, por lo que los atributos disponibles deben ser
transformados durante el preprocesado. Esto se puede realizar en la mayoría de
casos mediante simples métodos de minería de datos, como, por ejemplo, la
codificación de variables categóricas. Pero hay un problema, el archivo de
vídeo.

En este caso, una imagen es una matriz bidimensional donde cada celda contiene
una 3-tupla representando los valores rojo, verde y azul del pixel con el que se
corresponde. Un vídeo es una secuencia de imágenes junto con información
adicional, es este caso sólamente es relevante la tasa de refresco, que es la
frecuencia a la que ha sido capturada cada imagen. Esto es de gran importancia
para dar una componente temporal a las imágenes.

Con lo anterior se puede ver que un vídeo no se puede utilizar de forma directa,
sino que deberá pasar por una fase de preprocesado para ser transformado a un
conjunto de valores numéricos que lo describen.


