\capitulo{Técnicas y herraminetas}
\label{cha:Técnicas y herraminetas}

\section{Herramientas de desarrollo}

\subsection{Windows Subsystem for Linux}

Windows Subsystem for Linux (WSL) es una utilidad creada principalmente por
Microsoft y Canonical(Mantenedores de Ubuntu) que permite utilizar un sistema
operativo Linux desde Windows sin grandes pérdidas de rendimiento. Existen
varias opciones de distribución que se pueden utilizar (Kali Linux, OpenSUSE,
etc.), pero la mas utilizada y mejor mantenida para WSL es Ubuntu, por lo que es
la que se ha utilizado.

La mayor parte del desarrollo ha sido realizada desde WSL, donde se ha instalado
tanto \LaTeX como Python 3.9 y algunas dependencias, proceso mucho más simple de
realizar en Linux que en Windows.

\subsection{Visual Studio Code}

Visual Studio Code (VS Code) es un popular editor de texto altamente extensible
orientado al desarrollo creado por Microsoft. Gracias a su gran extensibilidad
es posible utilizarlo cómodamente en casi cualquier escenario. Mediante una de
estas extensiones (denominada literalmente WSL) se puede conectar con WSL y
utilizar los programas que tiene instalados como si hubiese sido ejecutado
diréctamente desde un sistema operativo Linux.

En este caso se ha utilizado VS Code para:

\begin{itemize}
    \item Crear tanto esta memoria como los anexos mediante la extensión
          \href{https://github.com/James-Yu/LaTeX-Workshop}{\LaTeX Workshop} que
          integra distintos compiladores de \LaTeX (pdfTeX, XeTeX, LuaTeX, etc.)
          dentro del programa y añade otras funcionalidades que hacen la edición
          de código \LaTeX mucho más conveniente.
    \item Crear la aplicación web, que utiliza el framework de JavaScript
          SvelteKit, mediante multitud de extensiones para integrar la multitud
          de librerías utilizadas (Svelte, TypeScript, Prettier, etc.).
    \item Editar archivos de configuración utilizados para la creación de los
          contenedores de Docker que componen la totalidad de la aplicación web
          (web, API y proxy inverso).
\end{itemize}

\subsection{PyCharm}

PyCharm es un Entorno de Desarrollo Integrado (IDE, por sus siglas en inglés)
creado y mantenido por JetBrains, una compañía que se dedica a crear software y
cuyos productos más representativos son multitud de IDEs para diferentes
lenguajes de programación (IntelliJ(Java), CLion(C y C++), GoLand(Go), etc.).

El objetivo de PyCharm es facilitar el desarrollo de aplicaciones que utilizad
Python y da soporte amplio tanto para los habituales ficheros \texttt{.py} como
para \texttt{.ipynb} (Notebooks de Jupyter), que han sido utilizados
extensamente en el proyecto.

Además, PyCharm por defecto permite la integración con entornos WSL de forma muy
fácil, simplemente se le debe indicar la localización del ejecutable de Python
en el sistema de archivos de la distribución de WSL.

En este proyecto PyCharm se ha utilizado para:

\begin{enumerate}
    \item Desarrollar la librería PADDEL, que contiene el código de Python común
    a los Notebooks de Jupyter utilizados durante la fase de investigación y a
    la aplicación web.
    \item Desarrollar la API que utiliza la aplicación web.
\end{enumerate}

\subsection{Git}

Actualmente Git es el sistema de control de versiones más extendido, en este
proyecto se ha usado para gestionar y guardar de forma segura los cambios que se
han ido realizado con el tiempo en la plataforma GitHub.

Aunque existen varias herramientas con interfaces gráficas que envuelven el
funcionamiento de Git (GitKraken, GitHub Desktop, etc.) para este proyecto se ha
utilizado diréctamente el comando \texttt{git} desde una terminal de WSL.
