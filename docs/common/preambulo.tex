\documentclass[a4paper,12pt,twoside]{memoir}

% Castellano
\usepackage[spanish,es-tabla,es-ucroman]{babel}
\selectlanguage{spanish}
\usepackage[T1]{fontenc}

% Fuente
\usepackage{lmodern}

% Imagenes
\usepackage{graphicx}
\graphicspath{{./img}}

% Utils
\usepackage{float}
\usepackage{microtype}
\usepackage{placeins}
\usepackage[table]{xcolor}
\usepackage{tcolorbox}
\usepackage{subfig}
\usepackage[shortlabels]{enumitem}
\usepackage{dirtree}
\usepackage{amsmath}
\usepackage{array}
\usepackage{multirow}

% Enlaces
\usepackage[colorlinks]{hyperref}
\hypersetup{
    allcolors = {red}
}

% Huérfanas y viudas
\widowpenalty100000
\clubpenalty100000

% Párrafos
\nonzeroparskip

% Capítulo
\chapterstyle{bianchi}

% Permitir mayúsculas con tilde en headers
\setlength{\headheight}{15.4pt}

% Variables
\title{Identificación de Parkinson por visión artificial}
\author{Catalin Andrei Cacuci}
\date{\today}
\newcommand{\thedni}{X7451927L}
\newcommand{\thetutor}{Alicia Olivares Gil}
\newcommand{\thecotutor}{Álvar Arnaiz González}

%%%%%%%%%%%%%%%%%%%%%%%%%%%%%%%%%%%%%%%%%%%%%%%%%%%%%%%%%%%%%%%%%%%%%%%%%%%%%%%%
%                                   COMANDOS
%%%%%%%%%%%%%%%%%%%%%%%%%%%%%%%%%%%%%%%%%%%%%%%%%%%%%%%%%%%%%%%%%%%%%%%%%%%%%%%%

% Imagen de cabecera
\newcommand{\cabecera}{\noindent\includegraphics[width=\textwidth]{cabecera}}

% Página en blanco que no incrementa contador
\newcommand{\blankpage}{
    \null
    \thispagestyle{empty}
    \addtocounter{page}{-1}
    \newpage
}

% Comando para insertar una imagen en un lugar concreto.
% Los parámetros son:
% 1 --> Ruta absoluta/relativa de la figura
% 2 --> Texto a pie de figura
% 3 --> Tamaño en tanto por uno relativo al ancho de página
\newcommand{\imagen}[3]{
    \begin{figure}[!h]
        \centering
        \includegraphics[width=#3\textwidth]{#1}
        \caption{#2}\label{fig:#1}
    \end{figure}
    \FloatBarrier
}

% Comando para insertar una imagen en un lugar concreto.
% Los parámetros son:
% 1 --> Ruta absoluta/relativa de la figura
% 2 --> Texto a pie de figura
% 3 --> Tamaño en tanto por uno relativo al ancho de página
\newcommand{\imagenConBorde}[3]{
    \begin{figure}[!h]
        \centering
        \fbox{\includegraphics[width=#3\textwidth]{#1}}
        \caption{#2}\label{fig:#1}
    \end{figure}
    \FloatBarrier
}

% Comando para insertar una imagen sin posición.
% Los parámetros son:
% 1 --> Ruta absoluta/relativa de la figura
% 2 --> Texto a pie de figura
% 3 --> Tamaño en tanto por uno relativo al ancho de página
\newcommand{\imagenflotante}[3]{
    \begin{figure}
        \centering
        \includegraphics[width=#3\textwidth]{#1}
        \caption{#2}\label{fig:#1}
    \end{figure}
}

% El comando \figuraConPosicion nos permite insertar figuras comodamente, y
% utilizando siempre el mismo formato. Los parametros son:
% 1 --> Porcentaje del ancho de página que ocupará la figura (de 0 a 1)
% 2 --> Fichero de la imagen
% 3 --> Texto a pie de imagen
% 4 --> Etiqueta (label) para referencias
% 5 --> Opciones que queramos pasarle al \includegraphics
% 6 --> Opciones de posicionamiento a pasarle a \begin{figure}
\newcommand{\figuraConPosicion}[6]{
    \setlength{\anchoFloat}{#1\textwidth}
    \addtolength{\anchoFloat}{-4\fboxsep}
    \setlength{\anchoFigura}{\anchoFloat}
    \begin{figure}[#6]
        \begin{center}
            \Ovalbox{
                \begin{minipage}{\anchoFloat}
                    \begin{center}
                        \includegraphics[width=\anchoFigura,#5]{#2}
                        \caption{#3}
                        \label{#4}
                    \end{center}
                \end{minipage}
            }
        \end{center}
    \end{figure}
}

% Comando para incluir imágenes en formato apaisado (sin marco).
\newcommand{\figuraApaisadaSinMarco}[5]{%
    \begin{figure}%
        \begin{center}%
            \includegraphics[angle=90,height=#1\textheight,#5]{#2}%
            \caption{#3}%
            \label{#4}%
        \end{center}%
    \end{figure}%
}

% Para las tablas
\newcommand{\otoprule}{\midrule [\heavyrulewidth]}

% Nuevo comando para tablas pequeñas (menos de una página).
\newcommand{\tablaSmall}[5]{%
    \begin{table}[H]
        \begin{center}
            \rowcolors {2}{gray!35}{}
            \begin{tabular}{#2}
                \toprule
                #4
                \otoprule
                #5
                \bottomrule
            \end{tabular}
            \caption{#1}
            \label{tabla:#3}
        \end{center}
    \end{table}
}

% Nuevo comando para tablas pequeñas (menos de una página).
\newcommand{\tablaSmallSinColores}[5]{%
    \begin{table}[H]
        \begin{center}
            \begin{tabular}{#2}
                \toprule
                #4
                \otoprule
                #5
                \bottomrule
            \end{tabular}
            \caption{#1}
            \label{tabla:#3}
        \end{center}
    \end{table}
}

\newcommand{\tablaApaisadaSmall}[5]{%
    \begin{landscape}
        \begin{table}
            \begin{center}
                \rowcolors {2}{gray!35}{}
                \begin{tabular}{#2}
                    \toprule
                    #4
                    \otoprule
                    #5
                    \bottomrule
                \end{tabular}
                \caption{#1}
                \label{tabla:#3}
            \end{center}
        \end{table}
    \end{landscape}
}

% Nuevo comando para tablas grandes con cabecera y filas alternas coloreadas en
% gris.
\newcommand{\tabla}[6]{%
    \begin{center}
        \tablefirsthead{
            \toprule
            #5
            \otoprule
        }
        \tablehead{
            \multicolumn{#3}{l}{\small\sl continúa desde la página anterior}\\
            \toprule
            #5
            \otoprule
        }
        \tabletail{
            \hline
            \multicolumn{#3}{r}{\small\sl continúa en la página siguiente}\\
        }
        \tablelasttail{
            \hline
        }
        \bottomcaption{#1}
        \rowcolors {2}{gray!35}{}
        \begin{xtabular}{#2}
            #6
            \bottomrule
        \end{xtabular}
        \label{tabla:#4}
    \end{center}
}

% Nuevo comando para tablas grandes con cabecera.
\newcommand{\tablaSinColores}[6]{%
    \begin{center}
        \tablefirsthead{
            \toprule
            #5
            \otoprule
        }
        \tablehead{
            \multicolumn{#3}{l}{\small\sl continúa desde la página anterior}\\
            \toprule
            #5
            \otoprule
        }
        \tabletail{
            \hline
            \multicolumn{#3}{r}{\small\sl continúa en la página siguiente}\\
        }
        \tablelasttail{
            \hline
        }
        \bottomcaption{#1}
        \begin{xtabular}{#2}
            #6
            \bottomrule
        \end{xtabular}
        \label{tabla:#4}
    \end{center}
}

% Nuevo comando para tablas grandes sin cabecera.
\newcommand{\tablaSinCabecera}[5]{%
    \begin{center}
        \tablefirsthead{
            \toprule
        }
        \tablehead{
            \multicolumn{#3}{l}{\small\sl continúa desde la página anterior}\\
            \hline
        }
        \tabletail{
            \hline
            \multicolumn{#3}{r}{\small\sl continúa en la página siguiente}\\
        }
        \tablelasttail{
            \hline
        }
        \bottomcaption{#1}
        \begin{xtabular}{#2}
            #5
            \bottomrule
        \end{xtabular}
        \label{tabla:#4}
    \end{center}
}

\definecolor{cgoLight}{HTML}{EEEEEE}
\definecolor{cgoExtralight}{HTML}{FFFFFF}

% Nuevo comando para tablas grandes sin cabecera.
\newcommand{\tablaSinCabeceraConBandas}[5]{%
    \begin{center}
        \tablefirsthead{
            \toprule
        }
        \tablehead{
            \multicolumn{#3}{l}{\small\sl continúa desde la página anterior}\\
            \hline
        }
        \tabletail{
            \hline
            \multicolumn{#3}{r}{\small\sl continúa en la página siguiente}\\
        }
        \tablelasttail{
            \hline
        }
        \bottomcaption{#1}
        \rowcolors[]{1}{cgoExtralight}{cgoLight}

        \begin{xtabular}{#2}
            #5
            \bottomrule
        \end{xtabular}
        \label{tabla:#4}
    \end{center}
}