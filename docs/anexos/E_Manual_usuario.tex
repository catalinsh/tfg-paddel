\capitulo{Documentación de usuario}
\label{cha:Documentación de usuario}

\section{Introducción}

Este apartado detalla la información que un usuario deberá tener para poder
utilizar el software creado en el proyecto. El apartado se divide en dos
secciones, una para la aplicación web y otra para el <<script>> de generación de
modelos.

\section{Aplicación web}

Aquí se detalla la forma en la que se utiliza la aplicación web, documentando
las diferentes funcionalidades disponibles.

\subsection{Requisitos de usuarios}

Para utilizar la aplicación el único requisito es disponer de una versión
reciente (a ser posible de los últimos dos años) de un navegador, programa que
suele venir por defecto en la mayoría de sistemas operativos. Además, cabe
destacar que la ejecución de JavaScript deberá estar permitida, aunque esto no
debería suponer un problema, al tener que ser desacivada de forma manual.

\subsection{Instalación}

Como la aplicación está alojada en un servidor remoto, el usuario no necesita
instalar ninguna dependencia específica.

\subsection{Manual del usuario}

Desde un navegador se puede acceder a la aplicación mediante la url
\href{https://paddle.catalin.sh}{paddel.catalin.sh}.

\subsubsection{Formulario predicción}

La página principal consiste en un formulario desde el que se pueden subir los
datos necesarios para realizar una predicción mediante los modelos:

\begin{itemize}
    \item Mano dominante
    \item Mano mostrada en el vídeo
    \item Edad
    \item Sexo
    \item Archivo de vídeo
\end{itemize}

\imagenConBorde{manual_usuario/01.png}{Formulario -- Página principal}{1}

Una vez introducidos los datos se puede hacer click en el botón <<Enviar>>, si
faltase algún dato el usuario es notificado por el navegador. Esto comenzará el
proceso de envío de los datos, debido al tamaño relativamente grande que tienen
los archivos de vídeo, este proceso puede ser de una duración bastante larga.
Debido a eso el usuario pasa momentaneamente a una pantalla de carga antes de
mostrar el resultado.

\imagenConBorde{manual_usuario/02.png}{Pantalla de carga formulario -- Subida}{1}

Cuando finalice la subida de los datos dará comienzo el procesado de los mismos
en el servidor, este, de nuevo, tiene una duración relativamente larga (entre 20
y 30 segundos), por lo que se informa al usuario para que no salga de la página
antes de tiempo.

\imagenConBorde{manual_usuario/03.png}{Pantalla de carga formulario -- Procesado}{1}

Cuando el proceso anterior finaliza el servidor responde con el resultado de la
predicción.

\imagenConBorde{manual_usuario/04.png}{Resultado predicción}{1}

En caso de que se haya producido algún error en la petición (problemas de red,
archivo de vídeo inválido, etc.) se notifica al usuario mediante una pantalla de
error.

\imagenConBorde{manual_usuario/05.png}{Error predicción}{1}

\subsubsection{Cambio de idioma}

Acceder a \href{https://paddle.catalin.sh}{paddel.catalin.sh} leerá el idioma
preferido del navegador usado y redirigirá a la alternativa más adecuada de
entre los idiomas disponibles (es-ES y en-GB).

El usuario puede controlar el idioma utilizado mediante el selector de idioma
localizado en la parte superior derecha de todas las páginas de la aplicación.

\imagenConBorde{manual_usuario/06.png}{Selector de idiomas}{0.2}

Este menú, además de cambiar el idioma en la sesión actual, utiliza una variable
en
\href{https://developer.mozilla.org/es/docs/Web/API/Window/localStorage}{almacenamiento
local} para \textit{recordar} el idioma seleccionado y utilizarlo en posteriores
visitas a la aplicación.

\subsubsection{Administración}

Para acceder al panel de administración se utiliza el botón <<Administración>>
de la página principal, si el usuario tiene una sesión válida abierta, será
redirigido a la página de administración de modelos, en caso contrario será
llevado a la página de inicio de sesión.

\imagenConBorde{manual_usuario/07.png}{Formulario de inicio de sesión}{0.6}

Desde esta página un usuario administrador puede identificarse introduciendo su
nombre de usuario y su contraseña.

En caso de que los datos introducidos sean incorrecto se le notifica con un
mensaje de error.

\imagenConBorde{manual_usuario/08.png}{Error de inicio de sesión}{0.6}

\paragraph{Administración de modelos}

En este panel se pueden ver y gestionar los modelos disponibles en la
aplicación.

\subparagraph{Seleccionar modelo}

Para seleccionar el modelo a utilizar para realizar predicciones se puede
utilizar el botón <<Seleccionar>> correspondiente.

\imagenConBorde{manual_usuario/09.png}{Selección de modelo}{1}

\subparagraph{Añadir modelo}

Añadir un nuevo modelo se puede realizar mediante el botón <<Añadir modelo>>,
que abre una ventana modal con un formulario para añadir la información del
nuevo modelo.

\imagenConBorde{manual_usuario/10.png}{Añadir modelo}{1}

\subparagraph{Eliminar modelo}

Se puede eliminar cualquier modelo que no esté seleccionado haciendo clic sobre
el botón <<Eliminar>>, esto abre una ventana modal de confirmación.

\imagenConBorde{manual_usuario/11.png}{Confirmación eliminar modelo}{1}

Una vez eliminado la lista de modelos se actualiza, reflejando la acción.

\paragraph{Administración de usuarios}

En este panel se pueden ver y gestionar los usuarios existentes en la
aplicación.

\subparagraph{Añadir usuario}

Añadir un nuevo usuario se puede realizar mediante el botón <<Añadir usuario>>,
que abre una ventana modal con un formulario para añadir la información del
nuevo usuario.

\imagenConBorde{manual_usuario/12.png}{Añadir usuario}{1}

\subparagraph{Eliminar usuario}

Al igual que para eliminar modelos, eliminar un usuario se realiza mediante el
botón <<Eliminar>> del usuario correspondiente.

\imagenConBorde{manual_usuario/13.png}{Confirmación eliminar usuario}{1}

\section{Script generación modelos}

Este <<script>>, localizado en el archivo \texttt{train.py} se utiliza para
generar modelos compatibles con el sistema creado, por lo que se pueden subir a
la aplicación web.

\subsection{Requisitos de usuario}

Este script requiere ser ejecutado con la versión 3.9 de Python. Para ello es
recomendable utilizar un entorno virtual, ya sea mediante utilidades como
\href{https://docs.conda.io/en/latest/}{Conda} o \textit{venv}. Por
ejemplo, para crear un entorno virtual mediante \textit{venv}, se utiliza:

\texttt{python -m venv venv}

Este comando genera un entorno virtual en el directorio \texttt{venv} en
relación a la ruta actual.

A continuación, se debe activar el entorno, la forma de realizar esto es
dependiente del sistema operativo:

\texttt{source ./venv/bin/activate \# Linux/MacOS} \\

\texttt{./venv/Scripts/activate.ps1 \# Windows con PowerShell} \\

\texttt{./venv/Scripts/activate.bat \# Windows con CMD} \\

El último paso antes de poder lanzar el <<script>> es instalar las dependencias.
Para ello se utiliza el siguiente comando desde la raíz del proyecto.

\texttt{pip install -r requirements.txt}

Además, dependiendo del sistema operativo, puede ser necesario instalar
\href{https://ffmpeg.org/}{FFmpeg}.

\subsection{Instalación}

El <<script>> no se instala en el sistema, una vez se han cumplido con los
requisitos de usuario se puede ejecutar directamente.

\subsection{Manual de usuario}


