\newcommand{\cu}[9]{
    \begin{table}[p]
        \centering
        \begin{tabularx}{\linewidth}{ p{0.21\columnwidth} p{0.71\columnwidth} }
            \toprule
            \textbf{CU-#1}      & \textbf{#2} \\
            \toprule
            \textbf{Versión}              & 1.0          \\
            \textbf{Autor}                & \theauthor  \\
            \textbf{Requisitos asociados} & #3          \\
            \textbf{Descripción}          & #4          \\
            \textbf{Precondición}         & #5          \\
            \textbf{Acciones}             & #6          \\
            \textbf{Postcondición}        & #7          \\
            \textbf{Excepciones}          & #8          \\
            \textbf{Importancia}          & #9          \\
            \bottomrule
        \end{tabularx}
        \caption{CU-#1 #2.}
    \end{table}
}


\capitulo{Especificación de Requisitos}
\label{cha:Especificación de Requisitos}

\section{Introducción}

Antes de comenzar a crear un programa es importante detallar las características
que deberá tener (quienes son los usuarios, qué acciones pueden realizar, de qué
manera deberá responder el programa, etc.). En este caso se ha seguido UML
(Unified Modelling Language) como estándar para definir estas características.

\section{Objetivos generales}

Durante la fase de investigación del proyecto se ha creado una multitud de
modelos de inteligencia artificial con diferentes grados de precisión para
detectar la enfermedad de Parkinson en un individuo. Estos modelos son
almacenados como archivos binarios y los conocimientos específicos de
programación requeridos para su uso hacen imposible que la persona media o
personal médico pueda obtener una predicción.

El objetivo principal de esta aplicación es implementar una interfaz intuitiva
para interactuar con los modelos sin la necesidad de que el usuario final altere
de ninguna forma su equipo informático.

Como objetivo secundario, la aplicación debe permitir a los usuarios registrados
seleccionar el modelo utilizado mediante la misma interfaz usada para realizar
las predicciones.

\section{Catálogo de requisitos}

\subsection{Requisitos funcionales}

\begin{enumerate}[start=1,label={\bfseries RF\arabic*:}]
    \item El sistema debe permitir diferenciar entre dos roles: Usuario estándar
          (que no ha iniciado sesión) y Administrador.

          \imagenConBorde{mockups/login.png}{Mockup de la página de inicio de sesión}{0.5}
          
    \item Cualquier usuario debe poder subir un archivo de vídeo junto con
          información adicional para ser procesados por un modelo de
          inteligencia artificial y obtener una predicción.

          \imagenConBorde{mockups/index.png}{Mockup de la página principal}{0.8}

    \item Los usuarios administradores deben poder iniciar sesión con su nombre
          de usuario y contraseña respectivos.
    \item Los usuarios administradores deben poder finalizar su sesión.
    \item Un inicio de sesión debe ser mantenido entre cambios de página y
          cierres del navegador durante 24 horas.
    \item Un administrador debe poder gestionar una lista con los modelos
          disponibles.

          \imagenConBorde{mockups/modelos.png}{Mockup de la página de gestión de modelos}{0.8}

    \item Un administrador debe poder añadir modelos a la lista de modelos
          disponibles.
    \item Un administrador debe poder eliminar modelos de la lista de modelos
          disponibles.
    \item Un administrador debe poder seleccionar de la lista de modelos
          disponibles aquel que se va a utilizar para realizar predicciones.
    \item Un administrador debe tener acceso a una lista con todos los usuarios
          registrados en la aplicación.

          \imagenConBorde{mockups/usuarios.png}{Mockup de la página de gestión de usuarios}{0.8}

    \item Un administrador debe poder añadir nuevos usuarios de tipo
          administrador a la aplicación, especificando su nombre de usuario y
          contraseña.
    \item Un administrador debe poder eliminar usuarios de la aplicación.
\end{enumerate}

\subsection{Requisitos no funcionales}

\begin{enumerate}[start=1,label={\bfseries RNF\arabic*:}]
    \item El tiempo desde que un usuario termina de subir el vídeo y sus datos
          hasta que recibe el resultado debe ser inferior a un minuto.
    \item El sistema debe ser fácil de usar, con una interfaz intuitiva que el
          usuario entienda de forma inmediata.
    \item La página web deberá utilizar etiquetas y atributos HTML de tal forma
          que se facilite la navegación para personas con accesibilidad
          limitada.
    \item La información de los usuarios se debe almacenar de forma segura,
          cifrando campos pertinentes como la contraseña.
\end{enumerate}

\section{Especificación de requisitos}

\subsection{Actores}

Existen dos tipos de actores en esta aplicación:

\begin{itemize}
    \item \textbf{Usuario Estándar}: Es aquel usuario que no ha iniciado sesión y
          tiene un acceso limitado a las capacidades del sistema.
    \item \textbf{Administrador}: Es aquel usuario que sí ha iniciado sesión en
          la aplicación y tiene acceso a todas las funcionalidades.
\end{itemize}

\subsection{Casos de uso}

Un caso de uso es la descripción de una interacción que el sistema debe permitir
realizar al usuario o usuarios pertinentes. El diagrama general de casos de uso
se puede observar en la figura \ref{fig:anexos/diagrama_casos_de_uso.pdf}

\imagen{anexos/diagrama_casos_de_uso.pdf}{Diagrama general de casos de uso}{1}

\cu{1}{Obtener predicción}
{RF2}
{Cualquier usuario debe poder obtener un predicción utilizado un modelo.}
{Estar en la página respectiva de la web.}
{
    \begin{enumerate}
        \def\labelenumi{\arabic{enumi}.}
        \tightlist
        \item El usuario selecciona el archivo de vídeo sobre el que desea
              recibir una predicción.
        \item El usuario introduce la información adicional necesaria.
        \item El usuario hace clic sobre el botón para subir la información.
        \item Se muestra una pantalla que informa al usuario sobre el estado de
              su petición.
        \item Se muestra la predicción obtenida.
    \end{enumerate}
}
{}
{
    \begin{itemize}
        \item [1] No se pueden extraer fotogramas con poses suficientes.
        \item [2] La información adicional introducida no es válida.
    \end{itemize}
}
{Alta}

\cu{2}{Iniciar sesión}
{RF1, RF3}
{Los usuarios administradores deben poder iniciar sesión con sus credenciales.}
{Estar en la página de inicio de sesión.}
{
    \begin{enumerate}
        \def\labelenumi{\arabic{enumi}.}
        \tightlist
        \item El usuario introduce su nombre de usuario.
        \item El usuario introduce su contraseña.
        \item El usuario hace clic en el botón de ``iniciar sesión''.
    \end{enumerate}
}
{El usuario es redirigido a la página de administración pertinente.}
{
    \begin{itemize}
        \item [3] Las credenciales introducidas no son válidas.
    \end{itemize}
}
{Alta}

\cu{3}{Finalizar sesión}
{RF1, RF4}
{Los usuarios administradores deben poder finalizar su sesión.}
{Tener una sesión activa.}
{
    \begin{enumerate}
        \def\labelenumi{\arabic{enumi}.}
        \tightlist
        \item El usuario hace clic en el botón para cerrar su sesión.
        \item La sesión se cierra y el usuario es redirigido a la página principal.
    \end{enumerate}
}
{}
{}
{Media}

\cu{4.1}{Ver lista de modelos}
{RF1, RF6}
{Los usuarios administradores deben poder ver la lista con los modelos disponibles.}
{}
{
    \begin{enumerate}
        \def\labelenumi{\arabic{enumi}.}
        \tightlist
        \item El usuario accede a la página de gestión de modelos.
        \item Se muestra una lista con los modelos existentes junto con algunas acciones.
    \end{enumerate}
}
{}
{
    \begin{itemize}
        \item [1] Usuario no ha iniciado sesión.
    \end{itemize}
}
{Alta}

\cu{4.2}{Seleccionar modelo}
{RF1, RF9}
{Los usuarios administradores deben poder seleccionar el modelo a usar para realizar predicciones.}
{El usuario debe estar en la lista de modelos.}
{
    \begin{enumerate}
        \def\labelenumi{\arabic{enumi}.}
        \tightlist
        \item El usuario hace clic en el botón para seleccionar un modelo
              concreto de la lista de modelos.
        \item Se informa al usuario de que el modelo ha sido seleccionado y es
              el que se va a utilizar para realizar predicciones.
    \end{enumerate}
}
{
    La base de datos cambia para establecer el modelo respectivo como seleccionado.
}
{
    \begin{itemize}
        \item [1] Usuario no ha iniciado sesión.
    \end{itemize}
}
{Alta}

\cu{4.3}{Añadir modelo}
{RF1, RF7}
{Los usuarios administradores deben poder añadir modelos.}
{El usuario debe estar en la lista de modelos.}
{
    \begin{enumerate}
        \def\labelenumi{\arabic{enumi}.}
        \tightlist
        \item El usuario hace clic en el botón para añadir un modelo.
        \item Se abre un formulario donde se pide el archivo del modelo junto
              con información adicional que describa al modelo.
        \item El usuario rellena el formulario y sube el archivo.
        \item Se muestra una página de carga mientras se procesa la información.
        \item Se devuelve al usuario a la lista de modelos, donde el nuevo
              modelo debería estar presente.
    \end{enumerate}
}
{}
{
    \begin{itemize}
        \item [1] Usuario no ha iniciado sesión.
        \item [5] Modelo subido no válido.
    \end{itemize}
}
{Alta}

\cu{4.4}{Eliminar modelo}
{RF1, RF8}
{Los usuarios administradores deben poder eliminar modelos.}
{El usuario debe estar en la lista de modelos.}
{
    \begin{enumerate}
        \def\labelenumi{\arabic{enumi}.}
        \tightlist
        \item El usuario hace clic en el botón para eliminar un modelo en concreto.
        \item Se abre una ventana modal que pregunta al usuario si está seguro.
        \item Si está seguro, se refresca la página mostrando la información actualizada.
    \end{enumerate}
}
{
    El modelo ha sido eliminado del sistema donde estaba almacenado y su entrada en
    la base de datos ha sido eliminada.
}
{
    \begin{itemize}
        \item [1] Usuario no ha iniciado sesión.
    \end{itemize}
}
{Media}

\cu{5.1}{Ver lista de usuarios}
{RF1, RF6}
{Los usuarios administradores deben poder ver la lista con los usuarios existentes.}
{}
{
    \begin{enumerate}
        \def\labelenumi{\arabic{enumi}.}
        \tightlist
        \item El usuario accede a la página de gestión de usuarios.
        \item Se muestra una lista con los usuarios existentes junto con algunas acciones que se puede realizar sobre los mismos.
    \end{enumerate}
}
{}
{
    \begin{itemize}
        \item [1] Usuario no ha iniciado sesión.
    \end{itemize}
}
{Baja}

\cu{5.2}{Añadir usuario}
{RF1, RF7}
{Los usuarios administradores deben poder añadir nuevos usuarios al sistema.}
{El usuario debe estar en la página con la lista de usuarios.}
{
    \begin{enumerate}
        \def\labelenumi{\arabic{enumi}.}
        \tightlist
        \item El usuario hace clic en el botón para añadir un usuario.
        \item Se abre un formulario donde se pide el nombre de usuario junto con
        su contraseña.
        \item El usuario rellena el formulario y lo envía.
        \item Se devuelve al usuario a la lista de usuarios, donde el nuevo
              usuario debería estar presente.
    \end{enumerate}
}
{}
{
    \begin{itemize}
        \item [1] Usuario no ha iniciado sesión.
        \item [3] Nombre de usuario demasiado corto.
        \item [3] Contraseña demasiado corta.
    \end{itemize}
}
{Baja}

\cu{5.3}{Eliminar usuario}
{RF1, RF8}
{Los usuarios administradores deben poder eliminar usuarios.}
{El usuario debe estar en la lista de usuarios.}
{
    \begin{enumerate}
        \def\labelenumi{\arabic{enumi}.}
        \tightlist
        \item El usuario hace clic en el botón para eliminar un usuario concreto.
        \item Se abre una ventana modal que pregunta al usuario si está seguro.
        \item Si está seguro, se refresca la página mostrando la información actualizada.
    \end{enumerate}
}
{
    El usuario ha sido eliminado de la base de datos.
}
{
    \begin{itemize}
        \item [1] Usuario no ha iniciado sesión.
    \end{itemize}
}
{Baja}