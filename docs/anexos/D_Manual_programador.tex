\chapter{Documentación técnica de programación}
\label{cha:Documentación técnica de programación}

\section{Introducción}

Esta sección contiene toda la información que una persona externa debería tener
para poder trabajar con las diferentes partes de este projecto.


\section{Estructura de directorios}

Los directorios del proyecto siguen la siguiente estructura:

\begin{itemize}
      \item \textbf{app}: Contiene todo lo relacionado con los contenedores de
            Docker que conforman la aplicación web.
            \begin{itemize}
                  \item \textbf{api}: Contiene el código fuente de la API que
                        implementa el framework FastAPI de Python.
                  \item \textbf{web}: Contiene el código fuente de la página web, se
                        trata de una implementación SvelteKit, que es un framework
                        de JavaScript.
                  \item \textbf{proxy}: Contiene la configuración de Nginx para el
                        proxy inverso que se utiliza para acceder a la API y a la
                        web desde el exterior.
            \end{itemize}
            
      \item \textbf{docs}: Contiene el código fuente en \LaTeX de esta
            documentación.
            
      \item \textbf{paddel}: Contiene el código fuente de la librería PADDEL, que
            tiene aquellas utilidades empleadas en la fase de investigación para
            facilitar su uso dentro de la aplicación web. Se trata de un paquete
            de Python que se puede instalar mediante el comando \texttt{pip}.
\end{itemize}


\section{Manual del programador}

Este proyecto utiliza la utilidad GNU Make para guardar secuencias de comandos y
gestionar las dependencias que existen entre estas. Para esto se utiliza un
archivo denominado \textit{Makefile} en los directorios relevantes. En caso de
que el sistema utilizado no disponga de la herramienta Make, se pueden
simplemente consultar el archivo \textit{Makefile} para ver los comandos que
deberían ser utilizados.


\subsection{PADDEL}

La librería PADDEL dispone de un archivo \textit{pyproject.toml} en el que se
detalla la forma de instalación, dependencias y algunas configuraciones de
herramientas de desarrollo.

Se incluye, además, un archivo \textit{Makefile} que define los siguientes
comandos:

\begin{itemize}
      \item \texttt{make install-dependencies}: Instala la librería con las
            dependencias de desarrollo, utilizadas para acciones como formateado
            (\textit{black}) o comprobación de tipos (\textit{mypy}).
      \item \texttt{make lint}: Realiza varias comprobaciones sobre el código
            fuente para detectar problemas de formateado o tipado e informa al
            usuario en caso de que existan.
      \item \texttt{make format}: Realiza un formateado sobre el código fuente,
            eliminando sentencias \texttt{import} no utilizadas, reordenando estas
            sentencias \texttt{import} y formateando el código para darle un
            aspecto homogéneo.
\end{itemize}

Para el entorno de desarrollo es muy recomendable utilizar un entorno virtual
para evitar realizar cambios sobre la instalación de Python a nivel de sistema.
Existen varias herramientas que permiten realizar esto, como \textit{venv} o
\textit{conda}. Con instalaciones que no se van a detallar en este documento al
existir documentación extensa sobre el uso de ambas herramientas. Es importante
destacar que la librería se ha creado en la versión 3.10 de Python, por lo que
es recomendable utilizar esta versión para el entorno virtual.

Una vez creado el entorno virtual de Python, este se deberá activar de forma
acorde a la herramienta utilizada y, a continuación, se puede ejecutar el
comando \texttt{make install-dependencies}, en ausencia de la utilidad Make se
puede utilizar \texttt{pip install -e .[dev]}, que instala el paquete en modo
editable junto con las dependencias de desarrollo.

Con todo esto se puede utilizar el comando \texttt{jupyter notebook notebooks}
para lanzar lanzar una instancia del kernel IPython en la carpeta
\textit{notebooks} donde se encuentra el código utilizado durante la fase de
investigación, preprocesado y entrenamiendo de modelos.

Para ejecutar los notebooks de Jupyter se debe crear una carpeta
\texttt{/data/raw} en la raíz del repositorio donde introducir los vídeos con el
formato de nombre adecuado.


\subsection{Aplicación}

Los componentes de la aplicación web se encuentran en el directorio \texttt{app}
se trata de tres contenedores de Docker, api, web y proxy. Para lanzar los
contenedores en modo desarrollo, es decir, que se detecten los cambios en la
mayoría de ficheros fuente y se actualizen los contenedores acordemente, se
puede utilizar \texttt{make docker-up-dev}, siempre que el motor de Docker esté
inicializado.

Para parar los contenedores se utiliza \texttt{make docker-down}, si, además, se
desea liberar memoria y eliminar las imágenes creadas se utiliza \texttt{make
      docker-clean}.

La edición de ficheros fuente se puede realizar diréctamente desde el sistema
anfitrión, pero, si se desea de ayudas como autocompletado y recomendaciones
será necesario utilizar un entorno de desarrollo que pueda interactuar con
contenedores, como por ejemplo \textit{Visual Studio Code} con la extensión
\textit{Dev Containers}.


\section{Compilación, instalación y ejecución del proyecto}

Para desplegar el proyecto simplemente se debe ejecutar \texttt{make docker-up}
desde la carpeta \texttt{app} con el motor de Docker inicializado, lo que
inicializará la aplicación web en el equipo anfitrión.


\section{Pruebas del sistema}
