\chapter{Plan de Proyecto Software}
\label{cha:Plan de Proyecto Software}

\section{Introducción}

Este anexo detalla el procedimiento que se ha seguido para gestionar el progreso
del proyecto y la realización de tareas.

\section{Planificación temporal}

La planificación del proyecto se llevará a cabo mediante una metodología ágil,
en concreto Scrum, con el apoyo de la herramienta
\href{https://www.zenhub.com/}{ZenHub} para la gestión de tareas y sprints
dentro de GitHub. En este caso cada sprint tiene una duración de una semana.

\subsection{Preparación}

Antes de comenzar el sprint 1 se realizaron las siguientes tareas:

\begin{itemize}
    \item Organizar el proyecto en diferentes carpetas con las partes que
          lo componen.
    \item Implementar la conectividad básica entre los contenedores de Docker
          que implementan la aplicación web.
    \item Implementar el parseado de los nombres de archivo de los vídeos.
    \item Implementar la extracción de puntos del esqueleto de la mano mediante
          Mediapipe Hands.
    \item Crear un paquete instalable o librería de Python con las utilidades
          que se utilizan en la fase de investigación para facilitar su uso
          posterior en la aplicación web.
\end{itemize}

\subsection{Sprint 1 (6-2-2023  12-2-2023)}

\begin{itemize}
    \item Añadir extracción de frecuencia de toques a partir de la secuencia
          de poses de la mano extraida por Mediapipe.
    \item Añadir extracción de diferencia entre la frecuencia de toques del
          intervalo de tiempo inicial y final de la secuencia de poses.
    \item Establecer ángulo máximo para la detección de toques.
    \item Cambiar la configuración de los contenedores de Docker para usar
          multietapa, separando la configuración para los entornos de desarrollo y
          producción.
\end{itemize}


\subsection{Sprint 2 (13-2-2023  19-2-2023)}

\begin{itemize}
    \item Añadir extracción de amplitud media.
    \item Añadir extracción de diferencia de amplitud media entre intervalo
          inicial y final del vídeo.
    \item Añadir extracción de variación de amplitud.
    \item Añadir extracción de velocidad del movimiento.
    \item Comenzar manual del programador.
    \item Reemplazar Nginx por Caddy como proxy inverso de la aplicación.
    \item Actualizar fichero README.md.
\end{itemize}

\subsection{Sprint 3 (20-2-2023  26-2-2023)}

\begin{itemize}
    \item Añadir extracción de características mediante TSFresh.
    \item Cambiar secuencia de instalación del contenedor de Docker para la
          API mejorando el cacheado de los pasos.
\end{itemize}

\subsection{Sprint 4 (27-2-2023  5-3-2023)}

\begin{itemize}
    \item Añadir herramientas de desarrollo a la memoria.
    \item Cambiar implementación de extracción de características de trabajos
          previos para que sea compatible con la librería TSFresh.
\end{itemize}

\subsection{Sprint 5 (6-3-2023  12-3-2023)}

\begin{itemize}
    \item Actualizar sección de herramientas de desarrollo.
\end{itemize}

\subsection{Sprint 6 (13-3-2023  19-3-2023)}

\begin{itemize}
    \item Implementar obtimización de hiperparámetros para varios modelos
          mediante \textit{grid search}.
    \item Añadir información temporal a la extracción de poses de Mediapipe a
          partir de la tasa de fotogramas.
\end{itemize}

\subsection{Sprint 7 (20-3-2023  26-3-2023)}

\begin{itemize}
    \item Añadir Perceptrón multicapa, Adaboost y XGBoost a los modelos de la
          optimización de hiperparámetros.
    \item Cambiar validación cruzada para utilizar 5 repeticiones de 2 \textit{folds}.
    \item Sustituir características de mano grabada y mano dominante por una
          única características, si está utilizando la mano dominante.
\end{itemize}

\subsection{Sprint 8 (27-3-2023  2-4-2023)}

\begin{itemize}
    \item Arreglar Makefile para la compilación de la documentación.
    \item Añadir generación de gráficas con los resultados obtenidos de la
          optimización de hiperparámetros mediante \textit{grid search}.
    \item Añadir selección del número de características a utilizar a la
          optimización de hiperparámetros.
    \item Refactorizar librería.
    \item Cambiar el framework de JavaScript de la web de SvelteKit a NextJS.
    \item Implementar una barra de navegación básica.
\end{itemize}


\section{Estudio de viabilidad}

\subsection{Viabilidad económica}

\subsection{Viabilidad legal}
