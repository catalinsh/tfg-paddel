\capitulo{Especificación de diseño}
\label{cha:Especificación de diseño}

\section{Introducción}

La especificación de diseño sirve como una guía para el proceso de diseño de la
aplicación, de modo que todas las personas involucradas en el proyecto saben
cómo debería ser el producto final y pueden comunicarse y colaborar de forma más
fácil.

En este anexo se intenta definir de forma clara el modo en que se van a
almacenar los datos, las distintas interacciones que deberán existir entre las
partes de la aplicación y la forma en la que el usuario final deberá interactuar
con ellas.

\section{Diseño de datos}

En este apartado se explica de forma detallada la forma en la que se almacenan
los datos con los que trabaja la aplicación (información de usuario, información
sobre los modelos y los archivos binarios de estos modelos).

En este caso se ha optado por utilizar el sistema gestor de bases de datos
PostgreSQL, esta decisión es debido a que ya existía cierta familiaridad con el
programa y a que es ampliamente soportado por SQLAlchemy, la librería ORM
(Object-Relational Mapping) utilizada en la API.

El diseño de datos creado es muy simple debido a que la aplicación en sí es muy
simple, su objetivo principal es permitir la fácil utilización de modelos
previamente entrenados para obtener predicciones y selección de los mismos por
parte de los administradores.

\section{Diseño procedimental}

\section{Diseño arquitectónico}

\imagen{mockups/index.png}{Mockup de la página principal}{0.8}

\imagen{mockups/modelos.png}{Mockup de la página de gestión de modelos}{0.8}
