\capitulo{Especificación de diseño}
\label{cha:Especificación de diseño}

\section{Introducción}

La especificación de diseño sirve como una guía para el proceso de diseño de la
aplicación, de modo que todas las personas involucradas en el proyecto saben
cómo debería ser el producto final y pueden comunicarse y colaborar de forma más
fácil.

En este anexo se intenta definir de forma clara el modo en que se van a
almacenar los datos, las distintas interacciones que deberán existir entre las
partes de la aplicación y la forma en la que el usuario final deberá interactuar
con ellas.

\section{Diseño de datos}

Esta sección especifica la forma en la que se han gestionado los datos usados
en el proyecto.

\subsection{Base de datos de la aplicación}

En este subapartado se explica de forma detallada la forma en la que se
almacenan los datos con los que trabaja la aplicación (información de usuario,
información sobre los modelos y los archivos binarios de estos modelos).

En este caso se ha optado por utilizar el sistema gestor de bases de datos
PostgreSQL. Esta decisión es debido a que ya existía cierta familiaridad con el
programa y a que es ampliamente soportado por SQLAlchemy, la librería ORM
(Object-Relational Mapping) utilizada en la API.

El diseño de datos creado es muy simple debido a que la aplicación en sí es muy
simple. Su objetivo principal es permitir la fácil utilización de modelos
previamente entrenados para obtener predicciones y selección de estos por parte
de los administradores.

En la base de datos se almacenan las siguientes entidades:

\begin{itemize}
    \item \textbf{Usuario (User)}: Corresponde con los usuarios registrados en
    la aplicación. Dispone de los campos \textit{id}, \textit{username}
    \textit{password} y \textit{removable}.
    \item \textbf{Modelo (Model)}: Corresponde con los modelos disponibles
    dentro de la aplicación. Dispone de los campos \textit{id}, \textit{name},
    \textit{path} y \textit{selected}.
\end{itemize}

Para determinar el modelo seleccionado se utiliza el campo \textit{selected},
que es de tipo booleano, puede valer \textit{null} y su valor es único. De modo
que este campo vale \textit{true} para el modelo seleccionado y \textit{null}
para todos los demás.

\section{Diseño procedimental}

\imagen{sequence/secuencia_prediccion.pdf}{Diagrama de secuencia para la obtención de una predicción}{1}

\section{Diseño arquitectónico}

El diseño arquitectónico de la aplicación fue seleccionado sin consultar ningún
patrón preexistente. Simplemente se tomaron las decisiones que parecían más
lógicas en su momento. Pese a esto, la aplicación ha tomado una estructura muy
similar a un patrón arquitectónico de cuatro capas.

Los patrones multicapa son de gran utilidad para reducir las dependencias
existentes entre las diferentes partes de la aplicación, aunque tienen la
desventaja que impactar el rendimiento durante la comunicación entre capas y
pueden provocar cambios en cascada. Una característica importante es que las
capas tienen un orden lógico y una capa solamente puede comunicarse con sus
capas adyacentes. Además, una capa solamente puede comenzar una petición hacia
su capa inferior, la comunicación hacia la capa superior se realiza siempre en
forma de respuesta.

Las siguientes capas conforman el patrón de cuatro capas, en orden de superior a
inferior.

\begin{enumerate}
    \item \textbf{Capa de presentación}: Determina la forma en que el usuario
    interactúa con la aplicación. En este caso corresponde con el frontend, es
    decir, la página web.
    \item \textbf{Capa de servicios}: Sirve como fachada entre la capa de
    presentación y la de lógica de negocios. En este caso corresponde con la API
    REST creada.
    \item \textbf{Capa de lógica de negocio}: Determina la forma en la que
    trabaja la aplicación con los datos para cumplir con los requisitos
    establecidos. En este caso corresponde con el código usado para realizar
    predicciones y gestionar modelos y usuario.
    \item \textbf{Capa de origen de datos}: Determina la comunicación con los
    sistemas que contienen los datos con los que trabaja la aplicación. En este
    caso es la base de datos en PostgreSQL.
\end{enumerate}

La figura \ref{fig:architecture/package.pdf} muestra las interacciones que
existen entre los diferentes componentes con conforman el proyecto.

\imagen{architecture/package.pdf}{Arquitectura de la aplicación}{1}

