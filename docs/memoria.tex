\documentclass[a4paper,12pt,twoside]{memoir}

% Castellano
\usepackage[spanish,es-tabla,es-ucroman]{babel}
\selectlanguage{spanish}
\usepackage[T1]{fontenc}

% Fuente
\usepackage{lmodern}

% Imagenes
\usepackage{graphicx}
\graphicspath{{../common/img/}{./img}}

% Utils
\usepackage{calc}
\usepackage{tcolorbox}
\usepackage{titling}
\usepackage{titlesec}

% Enlaces
\usepackage[colorlinks]{hyperref}
\hypersetup{
    allcolors = {red}
}

% Capítulo
\chapterstyle{bianchi}

% Párrafos
\nonzeroparskip

% Variables
\title{Detección del Parkinson}
\author{Catalin Andrei Cacuci}
\date{\today}
\newcommand{\thedni}{X7451927L}
\newcommand{\thetutor}{Álvar Arnaiz González}

%%%%%%%%%%%%
% Comandos %
%%%%%%%%%%%%

% Imagen de cabecera
\newcommand{\cabecera}{\noindent\includegraphics[width=\textwidth]{cabecera}}

% Añadir imagen
% [1] --> Tamaño en tanto por uno relativo al ancho de página
% 1 --> Ruta absoluta/relativa de la figura
% 2 --> Texto a pie de figura
\newcommand{\imagen}[3][0.9]{
    \begin{figure}[!h]
        \centering
        \includegraphics[width=#1\textwidth]{#2}
        \caption{#3}\label{fig:#2}
    \end{figure}
    \FloatBarrier
}

% Añadir imagen flotante
% [1] --> Tamaño en tanto por uno relativo al ancho de página
% 1 --> Ruta absoluta/relativa de la figura
% 2 --> Texto a pie de figura
\newcommand{\imagenflotante}[3][0.9]{
    \begin{figure}
        \centering
        \includegraphics[width=#1\textwidth]{#2}
        \caption{#3}\label{fig:#2}
    \end{figure}
}


% Capítulo
\newcommand{\capitulo}[1]{
    \stepcounter{chapter}
    \setcounter{section}{0}
    \setcounter{figure}{0}
    \setcounter{table}{0}
    \chapter*{\thechapter.\enskip #1}
    \addcontentsline{toc}{chapter}{\thechapter.\enskip #1}
    \markboth{#1}{#1}
}

\begin{document}

\renewcommand\maketitle{
    \thispagestyle{empty}

    \cabecera
    \vfill

    \begin{minipage}{0.74\textwidth}
        \begin{tcolorbox}[
                colback=blue!10,
                colframe=blue!20!black!40,
                top=16pt,
                bottom=16pt
            ]
            \begin{center}
                \large\textbf{TFG del Grado en Ingeniería Informática}

                \bigbreak

                \LARGE\textbf{\thetitle}
            \end{center}
        \end{tcolorbox}
    \end{minipage}
    \hfill
    \begin{minipage}{0.2\textwidth}
        \includegraphics[width=\textwidth]{escudoInfor}
    \end{minipage}


    \vfill

    \begin{center}
        \noindent\LARGE
        Presentado por \theauthor\\
        en Universidad de Burgos --- \thedate\\
        Tutora: \thetutor\\
        Cotutor: \thecotutor\\
    \end{center}

    \clearpage
}
\maketitle


\blankpage
\blankpage

% Tutor
\thispagestyle{empty}
\cabecera

\vfill

\noindent Dña. Alicia Olivares Gil y D. Álvar Arnaiz González, profesores
del departamento de Ingeniería Informática, área de Lenguajes y Sistemas
Informáticos.

\noindent Expones:

\noindent Que el alumno D. \theauthor, con DNI \thedni, ha realizado el
Trabajo final de Grado en Ingeniería Informática titulado Identificación de
Parkinson por visión artificial.

\noindent Y que dicho trabajo ha sido realizado por el alumno bajo la
dirección del que suscribe, en virtud de lo cual se autoriza su presentación
y defensa.

\begin{center}
    En Burgos, a {\large \today}
\end{center}

\vfill\vfill\vfill

% Author and supervisor
\begin{minipage}{0.45\textwidth}
    \begin{flushleft}
        Vº. Bº. de la tutora:\\[2cm]
        Dña. Alicia Olivares Gil
    \end{flushleft}
\end{minipage}
\hfill
\begin{minipage}{0.45\textwidth}
    \begin{flushleft}
        Vº. Bº. del cotutor:\\[2cm]
        D. Álvar Arnaiz González
    \end{flushleft}
\end{minipage}

\hfill
\vfill


\blankpage
\blankpage


\frontmatter


% Resumen
\renewcommand*\abstractname{Resumen}
\begin{abstract}
    La enfermedad de Parkinson afecta a cientos de miles de personas en España,
    cifra que algunas fuentes preveen que va a subir en el futuro. Aunque es una
    enfermedad incurable, una detección y tratamiento tempranos pueden mejorar
    la calidad de vida de los pacientes considerablemente.
    
    Este proyecto pretende utilizar técnicas de minería de datos y visión
    artificial para implementar un sistema capaz de detectar la presencia de
    bradicinesia, síntoma de la enfermedad de Párkinson, en un individuo de
    forma no intrusiva, mediante un simple vídeo.
\end{abstract}
\renewcommand*\abstractname{Descriptores}
\begin{abstract}
    enfermedad de Parkinson, bradicinesia, visión artificial, minería de datos,
    Docker, desarrollo web
\end{abstract}

\clearpage

% Resumen en inglés
\renewcommand*\abstractname{Abstract}
\begin{abstract}
    Parkinson's disease affects hundreds of thousands of people in Spain, this
    number is expected to increase in the future. Even though this illness has
    no cure, early detection and treatment can considerably improve the
    patient's quality of life.
    
    This project attempts to use data mining and computer vision techniques in
    order to detect bradykinesia, a common symptom of Parkinson's diseas, in a
    person in a non-intrusive way, with a simple video.
\end{abstract}
\renewcommand*\abstractname{Keywords}
\begin{abstract}
    Parkinson's disease, bradykinesia, computer vision, data mining, Docker, web
    development
\end{abstract}


\clearpage
\tableofcontents

\clearpage
\listoffigures

\clearpage
\listoftables



\mainmatter


\chapter{Introducción}
\label{cha:Introducción}

Descripción del contenido del trabajo y del estrucutra de la memoria y del
resto de materiales entregados.

% chapter Introducción (end)

\capitulo{Objetivos del proyecto}
\label{cha:Objetivos del proyecto}

El objetivo principal de este proyecto es la creación de un sistema basado en la
inteligencia artificial que permita la indentificación de la enfermedad de
Parkinson de forma consistente utilizando como entrada un vídeo de un gesto
específico (además de otra información), además este sistema debería ser
utilizable por la persona media. Para alcanzar este objetivo se plantean los
siguientes objetivos específicos:

\begin{enumerate}
    \item Revisar los trabajos previos relacionados para obtener unos
    conocimientos iniciales sobre el tema e identificar qué métodos funcionan
    mejor y peor.
    \item Diseñar e implementar una secuencia de pasos para extraer
    características relevantes de los vídeos para su uso en algoritmos de
    aprendizaje automático.
    \item Buscar un modelo que se ajuste considerablemente bien a los datos
    disponibles mediante alguna de las técnicas existentes en el campo de la
    inteligencia artificial para este propósito.
    \item Crear una aplicación web para interactuar de forma sencilla con los
    modelos y obtener predicciones. Debido al público de una aplicación de este
    tipo se debe tener muy en cuenta la accesibilidad a la hora de diseñarla.
    \item Añadir un panel de administración a la aplicación para actulizar el
    modelo utilizado para realizar las predicciones.
    \item Documentar el trabajo realizado.
\end{enumerate}

\capitulo{Conceptos teóricos}
\label{cha:Conceptos teóricos}

Este capítulo define algunos de los conceptos teóricos que se mencionan en este
documento.

\section{Minería de Datos}

La minería de datos es un área de la inteligencia artificial que consiste en el
diseño de algoritmos y modelos que permitan a los ordenadores aprender la regla
general que define un conjunto de datos sin ser explícitamente programados para
ello.

La minería de datos incluye, además de algoritmos de aprendizaje, todas las
metodologías y técnicas utilizadas para el tratamiento y filtrado del conjunto
de datos disponible.

\subsection{Preprocesado}

La información con la que se entrene un modelo de aprendizaje automático
determina en gran medida el rendimiento que podrá alcanzar, debido a esto es muy
frecuente realizar un paso previo a la creación del modelo, denominado
\textit{preprocesado} \cite{enwiki:1138293751}.

El objetivo del preprocesado es transformar la entrada de datos iniciales con el
fin de permitir y facilitar que un modelo se adapte a los mismos.

El preprocesado también tiene una gran influencia sobre el tiempo de computación
necesario para entrenar un modelo y sobre la complejidad que necesitará para
adaptarse a los datos.

El preprocesado suele componerse de algunos de los siguientes pasos.

\subsubsection{Extracción de características}

La extracción de características es el proceso de identificar, seleccionar y
transformar atributos relevantes de los datos de entrada para su uso en un
modelo. Por ejemplo, en este proyecto, se han extraido características como la
velocidad o amplitud de movimiento a partir de un vídeo.

Existen diversas técnicas para la extracción de características, incluyendo las
selección manual de características, o técnicas automatizadas como la reducción
de dimensionalidad, la extracción de características basada en redes neuronales
~\cite{intrator1991feature}, entre otras. La elección de la técnica de extracción
de características depende del conjunto de datos, del problema específico de
aprendizaje automático que se está abordando y del tipo de modelo de aprendizaje
automático que se está utilizando.

\subsubsection{Selección de características}

Los datos de entrada pueden ser muy complejos y estar compuestos por una gran
cantidad de información redundante o no relevante para el modelo. La selección
de características se utiliza para identificar  las características más
relevantes y representativas de los datos, que pueden ser utilizadas para
entrenar modelos de aprendizaje automático con mayor eficacia.

En la selección de características, se pueden utilizar técnicas como el análisis
de componentes principales (PCA)~\cite{mackiewicz1993principal}, el análisis
discriminante lineal (LDA)~\cite{xanthopoulos2013linear} o pruebas de
significancia, entre otras.

\subsubsection{Normalización}

Existen modelos muy sensibles a que existan diferencias en la escala de los
distintos atributos, como, por ejemplo, \textit{k-nearest neighbors}, por lo que
es muy habitual que en la fase de preprocesado se realize una normalización de
los datos, es decir, transformarlos de tal forma que utilicen la misma escala,
en general se suelen tomar valores en los intervalos $[0, 1]$ o $[-1, 1]$.

Aunque lo más habitual es que la normalización se haga sin distorsionar las
diferencias existentes entre los valores previos, existen situaciones en las que
puede ser ventajoso utilizar un método de normalización que sí altere estas
diferencias, por ejemplo, la normalización por cuantiles
~\cite{enwiki:1138433182}, en la que se modifican los valores para que sigan una
distribución normal, lo que consigue que, si existen valores muy lejanos a los
valores más comunes (\textit{outliers}), estos se acerquen al resto. 

\subsection{Aprendizaje supervisado}

Dentro del aprendizaje automático existen tres ramas o variantes dependiendo del
conjunto de datos del que se disponga, ya que estos datos son los que determinan
las técnicas, algoritmos y metodologías que se podrán utilizar, estas variantes
son el aprendizaje no supervisado, en el que los datos no están etiquetados, es
decir, el atributo que se desea predecir es desconocido (el ejemplo más
característico es el \textit{clustering}), aprendizaje semisupervisado, en el
que solo una parte de los datos están etiquetados, y el aprendizaje supervisado,
en el que el conjunto de datos está completamente etiquetado, este último es el
tipo de aprendizaje utilizado realizado en el proyecto y del que trata este
apartado.

El enfoque del aprendizaje supervisado consiste en la utilización de dos
conjuntos de datos, uno de entrenamiento (con datos etiquetados) y otro de test
(con datos no etiquetados), a continuación se utiliza un algoritmo de
aprendizaje para crear un modelo que <<aprenda>> de las instancias del conjunto
de entrenamiento una regla o procedimiento que le permita identificar los datos
del conjunto de test con la mayor precisión posible
\cite{learned2014introduction}.


\subsection{Clasificación}

En el ámbito del aprendizaje supervisado, clasificar instancias en categorías
predeterminadas es uno de los principales problemas a resolver.

En general, la resolución de un problema de clasificación se caracteriza por
delimitar qué zonas del espacio que contiene todas las instancias posibles
pertenecen a cada categoría, es decir, definir fronteras a partir de las cuales
las instancias pasan de una categoría a otra.

Existe una gran cantidad de métodos y algoritmos de clasificación que se pueden
utilizar para determinar estas fronteras, dependiendo del problema específico
que se intente resolver unos algoritmos se comportarán mejor que otros.

% TODO: Algunos ejemplos de algoritmos con alguna gráfica de clasificación con 2
% características.

\subsection{Sobreajuste}

Normalmente solo se dispone de una pequeña muestra para entrenar un modelo que
encuentre la regla general que define la población y se adapte lo mejor posible
a esta.

El problema de esto es que, si se utilizan todos los datos disponibles durante
el entrenamiento, no existe ninguna forma para verificar el comportamiento del
modelo cuando se encuentre con instancias que no ha visto antes. Esto hace
posible que el modelo ``memorice'' los datos con los que se ha entrenado pero no
generalice bien al encontrarse con datos nuevos. Esto se conoce como
sobreajuste.

% TODO: Añadir alguna gráfica.

Una solución muy común al sobreajuste es separar los datos disponibles en dos
grupos, un conjunto de entrenamiento y un conjunto de test o prueba. Durante el
entrenamiento el modelo solo tendrá acceso al conjunto de datos de
entrenamiento, mientras que el conjunto de test es utilizado para determinar el
rendimiento del modelo sobre datos nuevos mediante una de las métricas de
evaluación que se verán a continuación.

\subsection{Evaluación del modelo}

Para determinar si un modelo es mejor que otro es necesario definir alguna
métrica que asigne un valor numérico al rendimiento de cada modelo.

La métrica más intuitiva es lo que se conoce como precisión (\textit{accuracy}),
que es simplemente la proporción de predicciones (clasificaciones) acertadas del
total de predicciones realizadas.

Al igual que ocurren con el algoritmo de clasificación empleado, la métrica que
se use debe ser seleccionada en función del problema en cuestión. Una situación
muy común en la que utilizar la precisión no es lo más ideal es cuando se
trabaja con conjuntos de datos desbalanceados, es decir, cuando existen más
instancias de una clase que de otra.

Lo anterior es muy común en el ámbito médico cuando se intenta crear un modelo
que determine si un paciente padece una enfermedad concreta o no. En estas
situaciones se suele tener un grupo de control muy grande que no padece la
enfermedad y un grupo relativamente pequeño que sí la padece. Si, por ejemplo,
las proporciones de clases son 95\% y 5\% respectivamente, un modelo que siempre
prediga que el paciente no padece la enfermedad en cuestión obtendrá una
precisión del 95\%, lo que podría dar la falsa impresión de que se ajusta bien a
los datos, cuando, en realidad, es un modelo completamente inútil.

Algunas alternativas a la precisión incluyen:

\begin{itemize}
    \item \textbf{Sensibilidad}: Para una clase concreta, indica la capacidad
    del modelo de clasificar correctamente instancias de esa clase. En el caso
    médico esta métrica podría ser interesante para encontrar aquellos pacientes
    que sí padecen la enfermedad, aunque se clasifiquen mal algunos que no la
    padecen.
    \item \textbf{Especificidad}: Para una clase concreta, indica la capacidad
    del modelo para clasificar correctamente instancias que no pertenecen esa
    clase.
    \item \textbf{Valor-F1}: Es una medida que combina la sensibilidad y la
    especificidad mediante una media armónica.
\end{itemize}

\subsection{Fuga de información}

En el campo del aprendizaje automático las fugas de información se producen
cuando, de alguna forma, se utiliza información perteneciente al conjunto de
datos de test para entrenar un modelo, esto no significa únicamente utilizar
instancias de test durante el entrenamiento, las fugas de información se pueden
producir de forma mucho más sutil y ser difíciles de detectar. Por ejemplo, si
se utilizase el conjunto de test para seleccionar las características durante el
preprocesado se estaría produciendo una fuga de información.

Las fugas de información no son algo que debería ser siempre evitado, existen
situaciones en las que son necesarias dependiendo del caso concreto. Pero sí es
importante que sean detectadas y tenidas en cuenta a la hora de analizar los
resultados obtenidos, ya que, en caso contrario, se podría llegar a conclusiones
equívocas (demasiado optimistas por ejemplo).

\subsection{Validación cruzada}

En un apartado anterior se ha visto que para evitar el sobreajuste se puede
dividir la muestra de datos disponible en un conjunto de datos de entrenamiento
y uno de test. Al realizar esta división es posible que, en especial al trabajar
con muestras pequeñas, se realice una división que no sea útil para validar el
modelo. Por ejemplo, si por casualidad se seleccionase un conjunto de test
``fácil de predecir'' con el conjunto de entrenamiento dado, se medirá un
rendimiento mayor al real.

La solución a esto es la validación cruzada, que consiste en utilizar múltiples
pares de conjuntos de validación y test, de tal forma que se entrenen y validen
tantos modelos como cantidad de estos pares y cada instancia sea utilizada para
la validación al menos una vez, el resultado final de la validación es la media
de la métrica elegida de cada iteración.

Existen varios métodos para determinar el número de iteraciones a realizar y el
tamaño de los grupos de entrenamiento y test en cada iteración, por ejemplo:

\begin{itemize}
    \item \textbf{\textit{Leave-One-Out}}: Para una muestra de tamaño $N$,
    consiste en realizar $N$ iteraciones de entrenamiento y validación
    utilizando cada vez una única instancia para el conjunto de test. Es un
    método que se acerca bastante al rendimiento que se obtendría si se
    utilizase la muestra entera como conjunto de entrenamiento, pero tiene el
    problema de que se van a tener que realizar $N$ iteraciones, lo que implica
    un tiempo de computación necesario muy grande.
    \item \textbf{\textit{K-fold}}: Consite en dividir la muestra en conjuntos
    de entrenamiento y test de proporciones $(K-1)/K$ y $1/K$ respectivamente,
    realizando un total de $K$ iteraciones. Además se puede realizar varias
    repeticiones de este tipo de validación cruzada para reducir la variabilidad
    del error obtenido.
\end{itemize}

\subsection{Optimización de hiperparámetros}

Los algoritmos de clasificación suelen tener parámetros que determinan la forma
en la que se van a ajustar a los datos, por ejemplo, la tasa de aprendizaje o el
número de iteraciones máximas, estos valores se denominan hiperparámetros.

El proceso de búsqueda dentro del espacio que contiene todas las combinaciones
posibles de hiperparámetros para un algoritmo concreto se denomina optimización
de hiperparámetros.

En general, la optimización de hiperparámetros consiste en probar diferentes
combinaciones de valores y realizar el proceso de entrenamiento y test mediante
validación cruzada con una métrica apropiado hasta dar con la combinación más
adecuada. Existen varias formas para realizar esto, como, por ejemplo:

\begin{itemize}
    \item \textbf{\textit{Grid search}}: Proceso mediante el cual se definen los
    valores posibles que puede tomar cada parámetro y se prueba cada combinación
    de estos valores hasta encontral la mejor. Es una búsqueda exhaustiva que
    puede dar lugar a una gran cantidad de combinaciones, lo que implica un
    tiempo de computación muy grande.
    \item \textbf{\textit{Randomized search}}: Proceso similar al anterior en el
    que se especifican los valores posibles para cada parámetro, pero en este
    caso se realiza la búsqueda de forma aleatoria de forma que se limita el
    tiempo de computación necesario. Se puede utilizar para obtener una vista
    general del espacio de combinaciones de parámetros para realizar
    posteriormente una búsqueda más exhaustiva en un subespacio más pequeño.
    \item \textbf{\textit{Técnicas de optimización}}: Existen diversos métodos
    que intentan determinar la forma que tiene el rendimiento del modelo con
    respecto al espacio de combinaciones de parámetros posibles utilizando
    iteraciones anteriores para buscar los valores óptimos sin tener que probar
    todas las combinaciones como ocurre con \textit{grid search}. Un ejemplo de
    esto es la Optimización Bayesiana \cite{wu2019hyperparameter} en la que se
    intenta predecir la forma que toma la métrica de evaluación en función de
    los hiperparámetros elegidos, y así realizar una búsqueda más efectiva.
\end{itemize}

\capitulo{Técnicas y herraminetas}
\label{cha:Técnicas y herraminetas}

\section{Herramientas de desarrollo}

\subsection{Windows Subsystem for Linux}

Windows Subsystem for Linux (WSL) es una utilidad creada principalmente por
Microsoft y Canonical(Mantenedores de Ubuntu) que permite utilizar un sistema
operativo Linux desde Windows sin grandes pérdidas de rendimiento. Existen
varias opciones de distribución que se pueden utilizar (Kali Linux, OpenSUSE,
etc.), pero la mas utilizada y mejor mantenida para WSL es Ubuntu, por lo que es
la que se ha utilizado.

La mayor parte del desarrollo ha sido realizada desde WSL, donde se ha instalado
tanto \LaTeX como Python 3.9 y algunas dependencias, proceso mucho más simple de
realizar en Linux que en Windows.

\subsection{Visual Studio Code}

Visual Studio Code (VS Code) es un popular editor de texto altamente extensible
orientado al desarrollo creado por Microsoft. Gracias a su gran extensibilidad
es posible utilizarlo cómodamente en casi cualquier escenario. Mediante una de
estas extensiones (denominada literalmente WSL) se puede conectar con WSL y
utilizar los programas que tiene instalados como si hubiese sido ejecutado
diréctamente desde un sistema operativo Linux.

En este caso se ha utilizado VS Code para:

\begin{itemize}
    \item Crear tanto esta memoria como los anexos mediante la extensión
          \href{https://github.com/James-Yu/LaTeX-Workshop}{\LaTeX Workshop} que
          integra distintos compiladores de \LaTeX (pdfTeX, XeTeX, LuaTeX, etc.)
          dentro del programa y añade otras funcionalidades que hacen la edición
          de código \LaTeX mucho más conveniente.
    \item Crear la aplicación web, que utiliza el framework de JavaScript
          SvelteKit, mediante multitud de extensiones para integrar la multitud
          de librerías utilizadas (Svelte, TypeScript, Prettier, etc.).
    \item Editar archivos de configuración utilizados para la creación de los
          contenedores de Docker que componen la totalidad de la aplicación web
          (web, API y proxy inverso).
\end{itemize}

\subsection{PyCharm}

PyCharm es un Entorno de Desarrollo Integrado (IDE, por sus siglas en inglés)
creado y mantenido por JetBrains, una compañía que se dedica a crear software y
cuyos productos más representativos son multitud de IDEs para diferentes
lenguajes de programación (IntelliJ(Java), CLion(C y C++), GoLand(Go), etc.).

El objetivo de PyCharm es facilitar el desarrollo de aplicaciones que utilizad
Python y da soporte amplio tanto para los habituales ficheros \texttt{.py} como
para \texttt{.ipynb} (Notebooks de Jupyter), que han sido utilizados
extensamente en el proyecto.

Además, PyCharm por defecto permite la integración con entornos WSL de forma muy
fácil, simplemente se le debe indicar la localización del ejecutable de Python
en el sistema de archivos de la distribución de WSL.

En este proyecto PyCharm se ha utilizado para:

\begin{enumerate}
    \item Desarrollar la librería PADDEL, que contiene el código de Python común
    a los Notebooks de Jupyter utilizados durante la fase de investigación y a
    la aplicación web.
    \item Desarrollar la API que utiliza la aplicación web.
\end{enumerate}

\subsection{Git}

Actualmente Git es el sistema de control de versiones más extendido, en este
proyecto se ha usado para gestionar y guardar de forma segura los cambios que se
han ido realizado con el tiempo en la plataforma GitHub.

Aunque existen varias herramientas con interfaces gráficas que envuelven el
funcionamiento de Git (GitKraken, GitHub Desktop, etc.) para este proyecto se ha
utilizado diréctamente el comando \texttt{git} desde una terminal de WSL.

\capitulo{Aspectos relevantes del desarrollo del proyecto}
\label{cha:Aspectos relevantes del desarrollo del proyecto}

Este apartado contiene algunos aspectos relevantes sobre el desarrollo del
proyecto, incluyendo las razones detrás de las decisiones tomadas durante este y
el impacto que dichas decisiones han tenido sobre los resultados obtenidos.

\section{Fase de experimentación}

El conjunto de datos disponible durante la fase de experimentación consisten en
158 muestras con los siguientes ``atributos'':

\begin{itemize}
    \item Si la persona correspondiente a la instancia padece la enfermedad de
    Parkinson. Es lo que se busca predecir.
    \item Archivo de vídeo de una mano de la persona realizando la prueba de
    <<finger-tapping>>.
    \item Mano que aparece en el vídeo (izquierda o derecha).
    \item Fecha en la que se ha tomado el vídeo (irrelevante).
    \item Edad de la persona.
    \item Sexo de la persona.
    \item Mano dominante de la persona.
\end{itemize}

Cabe destacar que se ha tomado una muestra de cada mano de cada persona. Por lo
que este conjunto de datos ha sido obtenido de 79 personas. Aún así, se ha
tomado cada mano como una instancia distinta con el objetivo de tener más
instancias con las que trabajar.

Además, se dispone de 69 instancias de personas que padecen la enfermedad de
Parkinson y 89 instancias de personas sanas, por lo que se está trabajando con
un conjunto de datos ligeramente desbalanceado. Esto deberá tenerse en cuenta
para que no influya en los resultados obtenidos, por ejemplo, seleccionando
métricas adecuadas.

\subsection{Procesado de vídeo}

El objetivo de la fase de experimentación es crear un sistema que tenga por
entrada los atributos antes mencionados (salvo el objetivo) y por salida si esos
atributos corresponden con una persona que padece la enfermedad de Parkinson
(posiblemente junto con el grado de confianza de la predicción).

Los algoritmos de aprendizaje automático típicamente están diseñados para
trabajar con atributos numéricos, por lo que los atributos disponibles deben ser
transformados durante el preprocesado. Esto se puede realizar en la mayoría de
casos mediante simples métodos de minería de datos, como, por ejemplo, la
codificación de variables categóricas. Pero hay un problema, el archivo de
vídeo.

En este caso, una imagen es una matriz bidimensional donde cada celda contiene
una 3-tupla representando los valores rojo, verde y azul del pixel con el que se
corresponde. Un vídeo es una secuencia de imágenes junto con información
adicional, es este caso sólamente es relevante la tasa de refresco, que es la
frecuencia a la que ha sido capturada cada imagen. Esto es de gran importancia
para dar una componente temporal al vídeo.

Con lo anterior se puede ver que un vídeo no se puede utilizar de forma directa,
sino que deberá pasar por una fase multietapa de preprocesado para ser
transformado a un conjunto de valores numéricos que lo describen.

\subsubsection{Extracción de <<landmarks>>}

Una <<landmark>> en el campo de la visión aritificial es un punto de referencia
que se corresponde con un objeto físico. El objetivo de esta fase es convertir
el vídeo de la prueba de <<finger-tapping>> a una secuencia o serie temporal de
conjuntos de <<landmarks>> los cuales representan la posición de la mano durante
un instante concreto.

El primer paso para lograr esto es leer las imágenes del archivo de vídeo, esto
ha sido logrado mediante la librería de <<bindings>> de OpenCV para Python.
OpenCV es una librería de uso general para realizar tareas relacionadas con la
visión artificial, y permite leer multitud de formatos de vídeo mediante un
flujo de imágenes.

Se ha tomado especial precaución para no cargar en memoria todas las imágenes
del vídeo, sino ir leyéndolas cuando son necesarias. Un archivo de vídeo ocupa
relativamente poco espacio en disco debido a que está codificado mediante un
<<codec>> determinado, lo que reduce el tamaño del archivo. Pero, para poder
trabajar con las imágenes del vídeo, este debe ser decodificado.

La magnitud de lo anterior se ve claramente con el siguiente ejemplo. Un vídeo
con una resolución de 1920x1080 a 30 fotogramas por segundo codificado mediante
el codec H264 que ocupa 50,7MB, decodificado, ocuparía en torno a los 3,7GB.

A continuación se deben extraer las <<landmarks>> de las imágenes leidas. Esto
ha sido logrado mediante la librería Mediapipe Hands \cite{zhang2020mediapipe},
que permite extraer los siguentes puntos a partir de una imágen:

\imagen{relevant_aspects/hand-landmarks.png}{Mediapipe Hands <<Landmarks>>}{1}

Debido a la perspectiva desde la que se han tomados los vídeos de la prueba de
<<finger-tapping>> hay algunas <<landmarks>> que quedan ocultas en varias
ocasiones, lo que hace su posición detectada poco fiable. Por lo que se desechan
las <<landmarks>> en el intervalo [9-17] para solo tomar en cuenta la parte
``frontal'' de la mano.

Mediapipe Hands ofrece la posibilidad de extraer <<landmarks>> de las imágenes
como si estas no tuviesen relación entre sí o realizar la extracción teniendo en
cuenta que se trata de un vídeo, así teniendo en cuenta resultados anteriores
para determinar de forma más precisa las <<landmarks>> del fotograma actual.
Esto tiene la gran desventaja de que hace imposible la paralelización de este
proceso (debido a que cada extracción depende de los resultados anteriores).

Lo anterior no tiene un gran efecto durante la fase de investigación, ya que sí
se puede paralelizar la extracción de <<landmarks>> de cada vídeo. Donde sí
tiene efecto es sobre la aplicación web, empeorando la experiencia de los
usuarios.

En última instancia se tomó la decisión de no usar paralelización, y así obtener
resultados más precisos. Una alternativa planteada fue realizar la extracción en
paralelo solamente en la aplicación, pero esto se descartó ya que esto podría
introducir inconsistencias y variables adicionales que deberían ser tomadas en
cuenta.

\subsubsection{Extracción de series temporales}

Tras el proceso descrito en el apartado anterior se dispone una secuencia de
datos similar a la siguiente:

\imagen{relevant_aspects/poses_series.png}{Serie temporal de <<landmarks>>}{1}

Cada fila representa la pose de la mano durante un fotograma concreto. Se puede
observar que el índice es una fecha completa en lugar de otras opciones más
convenientes como los segundos desde el inicio del vídeo o lo que se conoce como
<<timestamp>>. Esto es una imposición de TSFresh para poder realizar la
extracción de características teniendo en cuenta esta componente temporal.

Cada celda representa el punto en el espacio en el que se encuentra la
<<landmark>> correspondiente en un instante de tiempo.

Aunque estas series temporales son una simplificación muy grande de lo vídeos
iniciales, aún no son del todo útiles. Al ser información sobre la posición en
el espacio, es dependiente de la forma en la que ha sido tomada la grabación. En
un entorno controlado esto no sería mucho problema, pero en este caso, como se
desea hacer el uso del sistema accesible a cualquier persona sin necesitar
equipamiento especial, se deben buscar alternativas.

En este caso se ha decidido tomar el ángulo entre el ángulo entre la punta del
dedo pulgar, la muñeca y la punta del dedo índice. Esta medida es útil ya que no
depende de cómo se ha tomado el vídeo ni del tamaño de la mano de la persona.
Quedando como resultado una serie temporal como la siguiente:

\imagen{relevant_aspects/angle_series.png}{Serie temporal de ángulos}{0.5}

La unidad usada en estos ángulos es el radián.

\subsubsection{Extracción de características}

El último paso para convertir un archivo de vídeo a atributos útiles para
entrenar un modelo de aprendizaje es extraer características que describan la
serie temporal obtenida. Algunos de estos atributos han sido extraídos de forma
específica teniendo en cuenta este problema concreto, mientras que otros (la
mayoría) son atributos genéricos de series temporales.

\paragraph{Detección de toques y amplitudes}

Para varios atributos específicos al problema han sido utilizados los instantes
de tiempo en los que se producen toques (el pulgar y el índice tocan) y
amplitudes máximas en el movimiento de <<finger-tapping>>.

La detección de estos instantes ha sido realizada mediante una búsqueda de
máximos gracias a la función \texttt{find\_peaks} de la librería
\href{https://scipy.org/}{scipy}. Pero esta función por si sola no da muy buenos
resultados.

\imagen{relevant_aspects/raw_peaks.png}{Amplitudes}{0.7}

El eje horizontal representa el tiempo que ha transcurrido desde el inicio del
vídeo. El eje vertical representa el ángulo en radianes entre el dedo índice, la
muñeca y el pulgar.

Se puede apreciar que obtener todos los máximos locales no es la mejor opción,
debe incorporarse algún criterio para determinar si un máximo debería tenerse en
cuenta o no. La función \texttt{find\_peaks} tiene algunos parámetros para este
propósito.

Tras probar varias opciones y comprobar manualmente los resultados obtenidos
sobre diferentes vídeos, se determinó que la mejor opción es establecer la
prominencia mínima (parámetro \texttt{prominence} de find\_peaks) que debe tener
un máximo local para ser considerado.

La prominencia se define como la medida que determina cuánto destaca un máximo
con respecto a sus máximos más cercanos. Se probaron varios valores estáticos,
pero como cada vídeo tiene características diferentes era necesario utilizar
algo más dinámico. La desviación típica se comportaba bastante bien, pero cuando
la amplitud variaba con el tiempo (síntoma de la enfermedad de Parkinson) los
máximos de menor magnitud no eran detectados.

Lo anterior fue solucionado  mediante el uso de una desviación típica móvil. Se
establece una ventana que determina la ``zona'' a la que pertenece un valor y se
utiliza la desviación típica de esa ``zona'' como prominencia mínima para que el
punto pueda ser considerado un máximo.

\imagen{relevant_aspects/rolling_std_peaks.png}{Amplitudes mediante desviación típica móvil}{0.7}

Se puede apreciar que los resultados son ampliamente mejores. Incluso cuando la
amplitud decrece con el tiempo:

\imagen{relevant_aspects/rolling_std_peaks_id.png}{Amplitudes mediante desviación típica móvil con la enfermedad de Parkinson}{0.7}

Aunque si el movimiento es errático, los resultados también lo son, como es de
esperar.

\imagen{relevant_aspects/rolling_std_peaks_erratic.png}{Amplitudes mediante desviación típica móvil con movimiento errático}{0.7}

Para detectar los toques (mínimos) es tan simple como utilizar la serie de
ángulos negada para detectar los puntos donde se encuentran los máximos. Además
se ha establecido un umbral para que solo se consideren toques si se encuentra
entre 0 y 0.1 radianes.

\imagen{relevant_aspects/rolling_std_valleys.png}{Amplitudes mediante desviación típica móvil con movimiento errático}{0.7}

\paragraph{Características detectadas}

A continuación se van a detallar algunas de las características extraídas de
mayor interés. Estos valores ya son atributos que describen el vídeo, por lo que
pueden ser utilizados para la generación de un modelo.

\begin{itemize}
    \item \textbf{Velocidad media del movimiento}: Una característica de la
    bradicinesia es la ralentización del movimiento. Esto va a ser reflejado en
    la velocidad media (en radianes por segundo) que tiene el movimiento
    \item \textbf{Frecuencia de amplitudes y toques}: Dos magnitudes que miden
    cuantos toques o amplitudes se realizan por segundo de media durante el
    movimineto.
    \item \textbf{Amplitud media}: Otra característica de la enfermedad de
    Párkinson es la debilidad, lo que se puede ver reflejado sobre la amplitud
    media del movimiento. Es simplemente la media de las amplitudes antes
    detectadas.
\end{itemize}

Además de las anteriores, se extrayeron 794 características genéricas
mediante la librería TSFresh.

\subsection{Limpieza de datos}

En este punto se dispone de un conjunto de 158 instancias y unos 800 atributos,
algunos de ellos categóricos, estos deberán ser codificados para que sean
atributos numéricos, con los que se trabaja mejor.

Además, se determinó que todas las instancias deberán tener un tiempo de
detección de la mano realizando la prueba de <<finger-tapping>> mínimo de 15
segundos para garantizar en cierto modo la precisión de los atributos extraidos
y poder compara instancias con el mínimo número de influencias sobre las que no
se tiene control.

Debido a la decisión anterior se pasó de disponer de 158 instancias a únicamente
156. Esto es desafortunado debido al ya pequeño número de muestras, pero se
consideró una pérdida que merece la pena con el fin de obtener mejores
resultados.

Por último se sustituyeron las característica <<mano dominante>> y <<mano que
aparece en el vídeo>> por una única característica <<es la mano que aparece en
el vídeo la dominante>> con el objetivo de reducir la conocida ``maldición de la
dimensionalidad'' al mismo tiempo que se mantiene la información relevante.

\capitulo{Trabajos relacionados}
\label{cha:Trabajos relacionados}

Durante la última década se ha intentado utilizar la visión por computador para
evaluar la Enfermedad de Parkinson en múltiples ocasiones con diferentes
resultados. Este capítulo recopila de forma resumida algunos de los trabajos más
relevantes.


\section{A computer vision framework for finger-tapping evaluation}

Este artículo \cite{khan2014computer} documenta el uso de visión por computador
para determinar el nivel de severidad de la Enfermedad de Parkinson y distinguir
entre individuos con esta enfermedad e individuos sin ella.

El método empleado se caracteriza por utilizar vídeos con la cara de la persona
y ambas manos a los lados de la cabeza, apuntando las puntas de los dedos hacia
la misma. Esto se utiliza para poder normalizar las distancias en base a
características faciales.

El estudio se realizó sobre 13 pacientes con la Enfermedad de Parkinson, tomando
17 vídeos de cada paciente durante un día, además de un grupo de control de 6
individuos, tomando 2 vídeos al día durante una semana por cada uno. Aunque
algunos de estos vídeos fueron descartados. En total se utilizaron 471 vídeos.


\paragraph{Metodología}

\begin{enumerate}
    \item Se detecta la cara del individuo para la normalización. Esto se basa
          en que la longitud de la mano de una persona adulta es aproximadamente
          igual a la altura de su cara.
    \item Se obtiene una serie temporal que representa la amplitud del
          movimiento de los dedos índice y pulgar de la mano dominante del
          individuo.
    \item Se extraen un total de 15 características de esta serie temporal, por
          ejemplo:
          \begin{itemize}
              \item Correlación cruzada media entre los máximos locales de dos
                    intervalos distintos de tiempo de la serie temporal. Esto
                    mismo se realiza también sobre los mínimos locales.
              \item Número total de toques de dedos durante la grabación.
              \item Velocidad media de la apertura de dedos.
              \item Velocidad media del cierre de dedos.
              \item \dots
          \end{itemize}
    \item Se realiza una selección de características eliminando aquellas
          redundantes y usando el algoritmo chi-cuadrado.
    \item Se entrena una máquina de vectores de soporte (SVM) mediante las
          características obtenidas para realizar la clasificación.
\end{enumerate}


\paragraph{Resultados}

En cuanto a la distinción entre pacientes de la Enfermedad de Parkinson y el
grupo de control se obtuvo una precisión del 95\%, que es una cifra que se
debería tomar con precaución debido que, aunque se han utilizado 471 vídeos,
estos provienen de únicamente 19 personas.


\section{The discerning eye of computer vision}

En este estudio \cite{williams2020discerning} realizado sobre 39 pacientes con
la Enfermedad de Parkinson y sobre un grupo de control de 30 individuos se
tomaron vídeos de ambas manos de cada individuo mientras realizan toques de los
dedos índice y pulgar (de forma similar a cómo se han tomado las muestras para
este trabajo). Dando un total de 133 vídeos (se descartó uno).

De estos vídeos se han extraído diferentes características y comprobado la
relación que existe entre éstas y diferentes escalas que clasifican el nivel de
gravedad de la Enfermedad en un paciente, como la \textit{Modified Bradykinesia
    Rating Scale} (MBRS) que categoriza el movimiento de los individuos en 5
niveles, del 0 al 4, siendo 0 un movimiento normal y 4 el nivel de mayor
gravedad.


\paragraph{Metodología}

\begin{enumerate}
    \item Se utiliza una librería de visión por computador, en concreto
          DeepCutLab, para obtener una serie temporal de la amplitud entre las
          puntas de los dedos pulgar e índice.
    \item Se normaliza esta serie temporal utilizando la amplitud máxima
          detectada, que va a convertirse en el valor 1, siendo todos los demás
          valores escalados proporcionalmente.
    \item Se extraen las siguientes características:
          \begin{itemize}
              \item Velocidad, calculada como la tasa media de cambio.
              \item Variabilidad de la amplitud, calculada como el coeficiente
                    de variación de la diferencia media entre máximos y mínimos
                    de diferentes intervalos de 1 segundo de la serie temporal.
              \item Regularidad del ritmo, calculada utilizando la Transformada
                    Rápida de Fourier y, a continuación, midiendo la potencia de
                    la frecuencia dominante más la potencia de las frecuencias
                    en un intervalo de 0.4 Hz alrededor de ésta (un ritmo más
                    regular concentra una mayor potencia en una única
                    frecuencia).
          \end{itemize}
\end{enumerate}


\paragraph{Resultados}

Se observó una correlación bastante alta entre las características utilizadas y
la categoría del individuo dentro de las escalas de medición de la Enfermedad de
Parkinson utilizadas medida por un experto en el campo.


\section{Supervised classification of bradykinesia}

Este estudio \cite{williams2020supervised} es muy similar al anteriormente
explicado, y está realizado por un equipo compuesto por casi los mismos
participantes. En este caso se utilizaron 70 vídeos, de ambas manos de 20
pacientes con la Enfermedad de Parkinson y de un grupo de control de 15
individuos.


\paragraph{Metodología}

La metodología es prácticamente igual que antes, la diferencia principal está en
las características que se extraen de la serie temporal correspondiente con la
amplitud, se ha obtenido:

\begin{itemize}
    \item Frecuencia, medida como la frecuencia máxima de la Transformada Rápida
          de Fourier de la serie temporal.
    \item Amplitud, calculada como la densidad espectral, que se ha obtenido
          mediante la integral cuadrada del espectro de la Transformada Rápida
          de Fourier.
\end{itemize}

Con estas características se ha realizado clasificación binaria mediante
clasificación bayesiana ingenua (naive bayes), regresión logística y máquina de
vectores de soporte, tanto con función lineal como con función de base radial.


\paragraph{Resultados}

Los mejores resultados se obtuvieron con máquina de vectores de soporte con
función de base radial, que coincide en un 73\% de los casos con la
clasificación de expertos en el campo.

\capitulo{Conclusiones y Líneas de trabajo futuras}
\label{cha:Conclusiones y Líneas de trabajo futuras}

En el campo de la minería de datos la fase de preprocesado tiene tanta
importancia como el propio entrenamiento de los modelos. En esta fase se
determina lo complejos que deberán ser los modelos para adaptarse a los datos y
el tiempo de entrenamiento que será necesario.

\section{Lineas de trabajo futuras}

El conjunto de datos utilizado contine muestras de personas que ya se conoce que
padecen la enfermedad de Parkinson, debido a esto, en este proyecto se ha
conseguido detectar la presencia de la enfermedad una vez está en un estado
avanzado. Esto no es muy útil para realizar una detección temprana y empezar un
tratamiento más efectivo en mermar los efectos de la enfermedad.

Una línea de trabajo futuro a largo plazo podría consistir en tomar vídeos de la
prueba de <<finger-tapping>> de forma aleatoria sobre la población, si la
muestra es lo suficientemente grande, algunas de esas personas desarrollarán la
enfermedad. Con esto podría ser posible crear modelos capaces de detectar
personas que van a padecer la enfermedad de Parkinson en el futuro, pudiendo
tomar las medidas precautorias necesarias con antelación.



\bibliographystyle{plain}
\bibliography{refs}

\end{document}
