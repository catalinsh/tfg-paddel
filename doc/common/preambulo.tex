\documentclass[a4paper,12pt,twoside]{memoir}

% Castellano
\usepackage[spanish,es-tabla,es-ucroman]{babel}
\selectlanguage{spanish}
\usepackage[T1]{fontenc}

% Fuente
\usepackage{lmodern}

% Imagenes
\usepackage{graphicx}
\graphicspath{{../common/img/}{./img}}

% Utils
\usepackage{calc}
\usepackage{tcolorbox}
\usepackage{titling}
\usepackage{titlesec}

% Enlaces
\usepackage[colorlinks]{hyperref}
\hypersetup{
    allcolors = {red}
}

% Capítulo
\chapterstyle{bianchi}

% Párrafos
\nonzeroparskip

% Variables
\title{Detección del Parkinson}
\author{Catalin Andrei Cacuci}
\date{\today}
\newcommand{\thedni}{X7451927L}
\newcommand{\thetutor}{Álvar Arnaiz González}

%%%%%%%%%%%%
% Comandos %
%%%%%%%%%%%%

% Imagen de cabecera
\newcommand{\cabecera}{\noindent\includegraphics[width=\textwidth]{cabecera}}

% Añadir imagen
% [1] --> Tamaño en tanto por uno relativo al ancho de página
% 1 --> Ruta absoluta/relativa de la figura
% 2 --> Texto a pie de figura
\newcommand{\imagen}[3][0.9]{
    \begin{figure}[!h]
        \centering
        \includegraphics[width=#1\textwidth]{#2}
        \caption{#3}\label{fig:#2}
    \end{figure}
    \FloatBarrier
}

% Añadir imagen flotante
% [1] --> Tamaño en tanto por uno relativo al ancho de página
% 1 --> Ruta absoluta/relativa de la figura
% 2 --> Texto a pie de figura
\newcommand{\imagenflotante}[3][0.9]{
    \begin{figure}
        \centering
        \includegraphics[width=#1\textwidth]{#2}
        \caption{#3}\label{fig:#2}
    \end{figure}
}
